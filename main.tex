\documentclass[a4paper]{article}

%% Language and font encodings
\usepackage[english]{babel}
\usepackage[utf8x]{inputenc}
\usepackage[T1]{fontenc}

%% Sets page size and margins
\usepackage[a4paper,top=3cm,bottom=2cm,left=3cm,right=3cm,marginparwidth=1.75cm]{geometry}

%% Useful packages
\usepackage{amsmath}
\usepackage{graphicx}
\usepackage[colorinlistoftodos]{todonotes}
\usepackage[colorlinks=true, allcolors=blue]{hyperref}

 
\title{Geometry of Perception}
\author{Lincoln A. Wallen }
\date{September 2019}

\begin{document}

\maketitle

\section{Introduction}
The ecological approach to perception starts with a reframing of the focus of study to the coupling of perception and action in an environment and, as such, provides a approach to a theory of intentionality via the dynamic, unfolding relationship between an organism going about its everyday business and the environment in which that business is situated \cite{Gibson1979}.  This is in contrast to vertical explanations via internal (mental) states of the organism that characterises much of cognitive science.  Such a horizontal respecification of intentionality is partially motivated by a desire to remain emphatically empirical, driven by observable phenomena, and eschew as far as is possible unwarranted theoretical speculation \cite{VanDijk2014}.

The organism-environment system itself is considered to rest vertically on an acceptable theory of dynamics that grounds the space of potential physical interactions within the system, involving (at minimum) mechanical, electromagnetic and gravitational phenomena.  While this assumption is shared by both ecological and established cognitive theories, the ecological point of departure is to seek a further, higher-level account, at the “natural scale" of each organism (class).  The assertion is that at this scale, the phenomenology of interactions within the system can be given a nomological treatment \cite{Mace2005}.

It is often suggested that Gibson and his followers reject physical notions of space and time (see \cite{Mace2005}).  It is clear from the above that this is not quite accurate; standard physical spacetime is a ground for the ecological theories, but is not considered the right level of description for the high-level ecological laws that are the object of study.  Our objective in this paper is to provide a geometro-dynamic account of ecological law, faithful to Gibson’s intent, but illustrating in what way it is compatible with standard physical dynamics. 

\bibliographystyle{alpha}
\bibliography{references}
\end{document}
